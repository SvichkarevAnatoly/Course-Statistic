\documentclass{article} % Класс печатного документа

\usepackage[utf8]{inputenc} % Кодировка исходного текста - utf8
\usepackage[english,russian]{babel} % Поддержка языка - русского с английским
\usepackage{indentfirst} % Отступ в первом абзаце
\usepackage{amsmath} % Для символа точек в выборке
\usepackage{xfrac}

\title{Лабораторная работа №2\\Ядерная оценка плотности распределения} % Заголовок документа
\author{Свичкарев А.\,В.\\группа 53601/4\\СПбПУ} % Автор документа
\date{\today} % Текущая дата

\begin{document} % Конец преамбулы, начало текста

\maketitle % Печатает заголовок, список авторов и дату

\section{Постановка задачи}
Дана выборка 100 элементов из стандартного нормального \mbox{распределения}:
\[x_1,x_2,\dotsc,x_{100}\sim\mathcal{N}(0,1)\]

Необходимо построить ядерные оценки плотности распределения, меняя значения параметра масштаба (bandwidth parameter) h:

\[\hat{f}_{n,h}(x) = \frac{1}{nh}K{\left(\frac{x-x_i}{h}\right)},\]

где $n$ -- размер выборки,\\
\indent $K$ -- гауссово ядро:

\[K(u) = \frac{1}{\sqrt{2\pi}}e^{\sfrac{-u^2}{2}}\]

\section{Заключение}
TODO2

\end{document} % Конец документа
