\documentclass{article} % Класс печатного документа

% для поддержки русского языка
\usepackage[T2A]{fontenc} % поддержка специальных русских символов
\usepackage[utf8]{inputenc} % Кодировка исходного текста - utf8
\usepackage[english,russian]{babel} % Поддержка языка - русского с английским
\usepackage{indentfirst} % Отступ в первом абзаце

\usepackage{hyperref} % Для вставки гиперссылок
\usepackage{listings} % Для вставки кусков кода
\usepackage{float} % Для точного позиционирования картинок

\title{Отчёт 1\\
Исследование критериев\\
Стьюдента, Колмогорова-Смирнова
и Уилкоксона (Манна-Уитни)} % заголовок документа
\author{Свичкарев А.\,В.} % Автор документа
\date{\today} % Текущая дата

\begin{document} % Конец преамбулы, начало текста

\maketitle % Печатает заголовок, список авторов и дату

\section*{Задачи}
\begin{enumerate}
    \item На основе критериев Стьюдента,
        Колмогорова-Смирнова и Уилкоксона (Манна-Уитни)
        проверить выполнение гипотезы $H_0$
        для различных объемов выборок: 10, 20, 100. 
    \item Сравнить критерии на эффективность работы,
        которую определим как близость полученного значения критерия
        к критическому табличному.
    \item Сравнить работу критериев при увеличении значения сдвига
        для второй выборки.
\end{enumerate}

\section*{Выполнение}

На следующих таблицах представлены результаты вычислений критериев:

\begin{table}[H]
\centering
\begin{tabular}{|c|c|c|c|c|c|}
\hline
\textbf{N} & \textbf{MeanShift} & \textbf{t} &
\textbf{Critical} & \textbf{$H_0$} & \textbf{Efficiency} \\ \hline
10  & 0.1       & 1.222103  & 2.100922 & TRUE  & 0.4183017  \\ \hline
10  & 0.5       & 0.7384197 & 2.100922 & TRUE  & 0.6485259  \\ \hline
10  & 1         & 1.702906  & 2.100922 & TRUE  & 0.1894485  \\ \hline
20  & 0.1       & 0.9361901 & 2.024394 & TRUE  & 0.5375455  \\ \hline
20  & 0.5       & 3.231181  & 2.024394 & FALSE & -0.5961227 \\ \hline
20  & 1         & 4.629824  & 2.024394 & FALSE & -1.287017  \\ \hline
100 & 0.1       & 1.023205  & 1.972017 & TRUE  & 0.4811381  \\ \hline
100 & 0.5       & 4.356737  & 1.972017 & FALSE & -1.209279  \\ \hline
100 & 1         & 6.878317  & 1.972017 & FALSE & -2.487959  \\ \hline
\end{tabular}
\caption{t-test}
\end{table}

\begin{table}[H]
\centering
\begin{tabular}{|c|c|c|c|c|c|}
\hline
\textbf{N} & \textbf{MeanShift} & \textbf{ks} &
\textbf{Critical} & \textbf{$H_0$} & \textbf{Efficiency} \\ \hline
10  & 0.1       & 0.5  & 0.40925  & FALSE & -0.2217471 \\ \hline
10  & 0.5       & 0.4  & 0.40925  & TRUE  & 0.02260232 \\ \hline
10  & 1         & 0.4  & 0.40925  & TRUE  & 0.02260232 \\ \hline
20  & 0.1       & 0.25 & 0.29408  & TRUE  & 0.1498912  \\ \hline
20  & 0.5       & 0.4  & 0.29408  & FALSE & -0.3601741 \\ \hline
20  & 1         & 0.55 & 0.29408  & FALSE & -0.8702394 \\ \hline
100 & 0.1       & 0.13 & 0.13403  & TRUE  & 0.0300679  \\ \hline
100 & 0.5       & 0.34 & 0.13403  & FALSE & -1.536746  \\ \hline
100 & 1         & 0.43 & 0.13403  & FALSE & -2.208237  \\ \hline
\end{tabular}
\caption{KS-test}
\end{table}

\begin{table}[H]
\centering
\label{my-label}
\begin{tabular}{|c|c|c|c|c|c|}
\hline
\textbf{N} & \textbf{MeanShift} & \textbf{u} &
\textbf{Critical} & \textbf{$H_0$} & \textbf{Efficiency} \\ \hline
10  & 0.1       & 34   & 32       & FALSE & -0.0625    \\ \hline
10  & 0.5       & 40   & 32       & FALSE & -0.25      \\ \hline
10  & 1         & 70   & 32       & FALSE & -1.1875    \\ \hline
20  & 0.1       & 214  & 151      & FALSE & -0.4172185 \\ \hline
20  & 0.5       & 300  & 151      & FALSE & -0.986755  \\ \hline
20  & 1         & 333  & 151      & FALSE & -1.205298  \\ \hline
100 & 0.1       & 5422 & 3000     & FALSE & -0.8073333 \\ \hline
100 & 0.5       & 6691 & 3000     & FALSE & -1.230333  \\ \hline
100 & 1         & 7542 & 3000     & FALSE & -1.514     \\ \hline
\end{tabular}
\caption{Wilcoxon-test}
\end{table}

\end{document} % Конец документа
