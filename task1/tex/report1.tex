\documentclass{article} % Класс печатного документа

\usepackage[utf8]{inputenc} % Кодировка исходного текста - utf8
\usepackage[english,russian]{babel} % Поддержка языка - русского с английским
\usepackage{indentfirst} % Отступ в первом абзаце
\usepackage{amsmath} % Для знака модуля

\title{Анализ мер положения} % Заголовок документа
\author{Свичкарев А.\,В.} % Автор документа
\date{\today} % Текущая дата

\begin{document} % Конец преамбулы, начало текста

\maketitle % Печатает заголовок, список авторов и дату

\section{Стандартное нормальное распределение}

Формула плотности:
\( f(x) = \frac{1}{\sqrt{ 2\pi }}e^{\frac{-x^2}{2}} \)

\begin{center}
	\begin{tabular}{|c| c|c|c|c|c|c|c| c|c|} \hline
		Номер выборки & 1 & 2 & 3 & 4 & 5 & 6 & 7 & \(\Sigma\) & Место \\ \hline
		\(\bar{x}\) & 1 & 3 & 3 & 2 & 2 & 2 & 2 & 15 & I \\ \hline
		\(medx\) & 3 & 4 & 1 & 4 & 3 & 1 & 1 & 17 & II-III \\ \hline
		\(z_R\) & 4 & 1 & 4 & 1 & 4 & 4 & 3 & 21 & IV \\ \hline
		\(z_Q\) & 2 & 2 & 2 & 3 & 1 & 3 & 4 & 17 & II-III \\ \hline
	\end{tabular}
\end{center}

Теоретическая ранжировка:

\[ D(\bar{x}) < D(z_Q) < D(medx) < D(z_R) \]

Наша ранжировка по выборкам совпадает с теоретической. В половине случаев \(z_R\) на последнем месте.

Оптимальность в статистике может быть только для множества выборок.

\section{Равномерное распределение}
Формула плотности:
\( \sigma(x) = \left\{
	\begin{array}{cl}
		\frac{1}{2\sqrt{3}}, & \mbox{если } \lvert x \rvert \leq \sqrt{3} \\
		0, & \mbox{если } \lvert x \rvert > \sqrt{3}  \\
\end{array} \right. \)

\end{document} % Конец документа
