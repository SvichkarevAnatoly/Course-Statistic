\documentclass{article} % Класс печатного документа

\usepackage[utf8]{inputenc} % Кодировка исходного текста - utf8
\usepackage[english,russian]{babel} % Поддержка языка - русского с английским
\usepackage{indentfirst} % Отступ в первом абзаце
\usepackage{amsmath} % Для знака модуля

\title{Анализ мер положения} % Заголовок документа
\author{Свичкарев А.\,В.} % Автор документа
\date{\today} % Текущая дата

\begin{document} % Конец преамбулы, начало текста

\maketitle % Печатает заголовок, список авторов и дату

\section{Стандартное нормальное распределение}

Формула плотности:
\( f(x) = \frac{1}{\sqrt{ 2\,\pi }}e^{\frac{-x^2}{2}} \)


Результаты ранжирования выборок среди группы:
\begin{center}
	\begin{tabular}{|c| c|c|c|c|c|c|c| c|c|} \hline
		Номер выборки & 1 & 2 & 3 & 4 & 5 & 6 & 7 & \(\Sigma\) & Место \\ \hline
		\(\bar{x}\)   & 1 & 3 & 3 & 2 & 2 & 2 & 2 & 15 & I \\ \hline
		\(med \, x\)      & 3 & 4 & 1 & 4 & 3 & 1 & 1 & 17 & II-III \\ \hline
		\(Z_R\)       & 4 & 1 & 4 & 1 & 4 & 4 & 3 & 21 & IV \\ \hline
		\(Z_Q\)       & 2 & 2 & 2 & 3 & 1 & 3 & 4 & 17 & II-III \\ \hline
	\end{tabular}
\end{center}

Теоретическая ранжировка:

\[ D(\bar{x}) < D(Z_Q) < D(med \, x) < D(Z_R) \]

Наша ранжировка по выборкам совпадает с теоретической. В половине случаев \(Z_R\) на последнем месте.

Оптимальность в статистике может быть только для множества выборок.

\section{Равномерное распределение}
Формула плотности:
\( \sigma(x) = \left\{
	\begin{array}{cl}
		\frac{1}{2 \, \sqrt{3}}, & \mbox{если } \lvert x \rvert \leq \sqrt{3} \\
		0, & \mbox{если } \lvert x \rvert > \sqrt{3}  \\
\end{array} \right. \)

Результаты ранжирования выборок среди группы:
\begin{center}
	\begin{tabular}{|c| c|c|c|c|c|c|c|c| c|c|} \hline
		Номер выборки & 1 & 2 & 3 & 4 & 5 & 6 & 7 & 8 & \(\Sigma\) & Место \\ \hline
		\(\bar{x}\)   & 1 & 3 & 3 & 3 & 3 & 2 & 3 & 1 & 19 & II \\ \hline
		\(med \, x\)      & 3 & 4 & 4 & 2 & 1 & 4 & 1 & 3 & 22 & III-IV \\ \hline
		\(Z_R\)       & 4 & 1 & 2 & 1 & 4 & 1 & 2 & 2 & 17 & I \\ \hline
		\(Z_Q\)       & 2 & 2 & 1 & 4 & 2 & 3 & 4 & 4 & 22 & III-IV \\ \hline
	\end{tabular}
\end{center}

\section{Загрязненное стандартное нормальное распределение}
\[ f(x) = 0.95 \cdot N(x;1) + 0.05 \cdot N(x;10)\]

Результаты ранжирования выборок среди группы:
\begin{center}
	\begin{tabular}{|c| c|c|c|c|c|c|c|c| c|c|} \hline
		Номер выборки & 1 & 2 & 3 & 4 & 5 & 6 & 7 & 8 & \(\Sigma\) & Место \\ \hline
		\(\bar{x}\)   & 3 & 3 & 3 & 3 & 3 & 1 & 2 & 2 & 20 & III \\ \hline
		\(med \, x\)      & 2 & 2 & 2 & 2 & 2 & 3 & 1 & 1 & 15 & I \\ \hline
		\(Z_R\)       & 4 & 4 & 1 & 4 & 4 & 4 & 4 & 4 & 29 & IV \\ \hline
		\(Z_Q\)       & 1 & 1 & 4 & 1 & 1 & 2 & 3 & 3 & 16 & II \\ \hline
	\end{tabular}
\end{center}

\section{Загрязненное стандартное нормальное распределение}
Должны получить аналогичные с Загрязнённым нормальным распределением результаты

\end{document} % Конец документа
