\documentclass{article} % Класс печатного документа

\usepackage[utf8]{inputenc} % Кодировка исходного текста - utf8
\usepackage[english,russian]{babel} % Поддержка языка - русского с английским
\usepackage{indentfirst} % Отступ в первом абзаце
\usepackage{amsmath} % Для знака модуля
\usepackage{tikz} % Для вставки графики из R

\title{Анализ мер положения} % Заголовок документа
\author{Свичкарев А.\,В.} % Автор документа
\date{\today} % Текущая дата

\begin{document} % Конец преамбулы, начало текста

\maketitle % Печатает заголовок, список авторов и дату

\section{Стандартное нормальное распределение}

Формула плотности:
\( f(x) = \frac{1}{\sqrt{ 2\,\pi }}e^{\frac{-x^2}{2}} \)

Результаты ранжирования выборок среди группы:
\begin{center}
	\begin{tabular}{|c| c|c|c|c|c|c|c| c|c|} \hline
		Номер выборки & 1 & 2 & 3 & 4 & 5 & 6 & 7 & \(\Sigma\) & Место \\ \hline
		\(\bar{x}\)   & 1 & 3 & 3 & 2 & 2 & 2 & 2 & 15 & I \\ \hline
		\(med \, x\)  & 3 & 4 & 1 & 4 & 3 & 1 & 1 & 17 & II-III \\ \hline
		\(Z_R\)       & 4 & 1 & 4 & 1 & 4 & 4 & 3 & 21 & IV \\ \hline
		\(Z_Q\)       & 2 & 2 & 2 & 3 & 1 & 3 & 4 & 17 & II-III \\ \hline
	\end{tabular}
\end{center}

Теоретическая ранжировка:
\[ D(\bar{x}) < D(Z_Q) < D(med \, x) < D(Z_R) \]

Наша ранжировка по выборкам совпадает с теоретической. В половине случаев \(Z_R\) на последнем месте.

Оптимальность в статистике может быть только для множества выборок.


\section{Равномерное распределение}
Формула плотности:
\( \sigma(x) = \left\{
	\begin{array}{cl}
		\frac{1}{2 \, \sqrt{3}}, & \mbox{если } \lvert x \rvert \leq \sqrt{3} \\
		0, & \mbox{если } \lvert x \rvert > \sqrt{3}  \\
\end{array} \right. \)

Результаты ранжирования мер по выборкам среди группы:
\begin{center}
	\begin{tabular}{|c| c|c|c|c|c|c|c|c| c|c|} \hline
		Номер выборки & 1 & 2 & 3 & 4 & 5 & 6 & 7 & 8 & \(\Sigma\) & Место \\ \hline
		\(\bar{x}\)   & 1 & 3 & 3 & 3 & 3 & 2 & 3 & 1 & 19 & II \\ \hline
		\(med \, x\)  & 3 & 4 & 4 & 2 & 1 & 4 & 1 & 3 & 22 & III-IV \\ \hline
		\(Z_R\)       & 4 & 1 & 2 & 1 & 4 & 1 & 2 & 2 & 17 & I \\ \hline
		\(Z_Q\)       & 2 & 2 & 1 & 4 & 2 & 3 & 4 & 4 & 22 & III-IV \\ \hline
	\end{tabular}
\end{center}

Теоретическая ранжировка:
\[ D(Z_R) < D(\bar{x}) < D(Z_Q) < D(med \, x) \]

Опять очень близко попали с практическим ранжированием.

\section{Загрязнённое стандартное нормальное распределение}
\[ f(x) = 0.95 \cdot N(x;1) + 0.05 \cdot N(x;10)\]

Результаты ранжирования выборок среди группы:
\begin{center}
	\begin{tabular}{|c| c|c|c|c|c|c|c|c| c|c|} \hline
		Номер выборки & 1 & 2 & 3 & 4 & 5 & 6 & 7 & 8 & \(\Sigma\) & Место \\ \hline
		\(\bar{x}\)   & 3 & 3 & 3 & 3 & 3 & 1 & 2 & 2 & 20 & III \\ \hline
		\(med \, x\)  & 2 & 2 & 2 & 2 & 2 & 3 & 1 & 1 & 15 & I \\ \hline
		\(Z_R\)       & 4 & 4 & 1 & 4 & 4 & 4 & 4 & 4 & 29 & IV \\ \hline
		\(Z_Q\)       & 1 & 1 & 4 & 1 & 1 & 2 & 3 & 3 & 16 & II \\ \hline
	\end{tabular}
\end{center}

Теоретическая ранжировка:
\[ D(med \, x) < D(Z_Q) < D(\bar{x}) < D(Z_R) \]

Результаты ранжировки мер среди группы повторяют теоретическое ранжирование.

\section{Загрязнённое равномерное распределение}
Ожидается получение аналогичных результатов Загрязнённому нормальному распределению.

Теоретическая ранжировка:
\[ D(med \, x) < D(Z_Q) < D(\bar{x}) < D(Z_R) \]

\section{Boxplot}
% Created by tikzDevice version 0.8.1 on 2015-10-20 22:25:39
% !TEX encoding = UTF-8 Unicode
\begin{tikzpicture}[x=1pt,y=1pt]
\definecolor{fillColor}{RGB}{255,255,255}
\path[use as bounding box,fill=fillColor,fill opacity=0.00] (0,0) rectangle (252.94,252.94);
\begin{scope}
\path[clip] ( 49.20, 61.20) rectangle (227.75,203.75);
\definecolor{drawColor}{RGB}{0,0,0}

\path[draw=drawColor,line width= 1.2pt,line join=round] (105.41,124.53) -- (171.54,124.53);

\path[draw=drawColor,line width= 0.4pt,dash pattern=on 4pt off 4pt ,line join=round,line cap=round] (138.47, 66.48) -- (138.47,105.52);

\path[draw=drawColor,line width= 0.4pt,dash pattern=on 4pt off 4pt ,line join=round,line cap=round] (138.47,198.47) -- (138.47,154.59);

\path[draw=drawColor,line width= 0.4pt,line join=round,line cap=round] (121.94, 66.48) -- (155.00, 66.48);

\path[draw=drawColor,line width= 0.4pt,line join=round,line cap=round] (121.94,198.47) -- (155.00,198.47);

\path[draw=drawColor,line width= 0.4pt,line join=round,line cap=round] (105.41,105.52) --
	(171.54,105.52) --
	(171.54,154.59) --
	(105.41,154.59) --
	(105.41,105.52);
\end{scope}
\begin{scope}
\path[clip] (  0.00,  0.00) rectangle (252.94,252.94);
\definecolor{drawColor}{RGB}{0,0,0}

\path[draw=drawColor,line width= 0.4pt,line join=round,line cap=round] ( 49.20, 93.78) -- ( 49.20,197.43);

\path[draw=drawColor,line width= 0.4pt,line join=round,line cap=round] ( 49.20, 93.78) -- ( 43.20, 93.78);

\path[draw=drawColor,line width= 0.4pt,line join=round,line cap=round] ( 49.20,128.33) -- ( 43.20,128.33);

\path[draw=drawColor,line width= 0.4pt,line join=round,line cap=round] ( 49.20,162.88) -- ( 43.20,162.88);

\path[draw=drawColor,line width= 0.4pt,line join=round,line cap=round] ( 49.20,197.43) -- ( 43.20,197.43);

\node[text=drawColor,rotate= 90.00,anchor=base,inner sep=0pt, outer sep=0pt, scale=  1.00] at ( 34.80, 93.78) {-1};

\node[text=drawColor,rotate= 90.00,anchor=base,inner sep=0pt, outer sep=0pt, scale=  1.00] at ( 34.80,128.33) {0};

\node[text=drawColor,rotate= 90.00,anchor=base,inner sep=0pt, outer sep=0pt, scale=  1.00] at ( 34.80,162.88) {1};

\node[text=drawColor,rotate= 90.00,anchor=base,inner sep=0pt, outer sep=0pt, scale=  1.00] at ( 34.80,197.43) {2};

\path[draw=drawColor,line width= 0.4pt,line join=round,line cap=round] ( 49.20, 61.20) --
	(227.75, 61.20) --
	(227.75,203.75) --
	( 49.20,203.75) --
	( 49.20, 61.20);
\end{scope}
\end{tikzpicture}


\section{Заключение}
От формы распределения сильно зависит выбор меры положения. Использование одной меры положения для различных распределений не приведёт к репрезентативным результатам.

\end{document} % Конец документа
